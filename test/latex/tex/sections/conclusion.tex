\documentclass[../main.tex]{subfiles}
\graphicspath{ {\subfix{../../img/}} }

\begin{document}
\section{Conclusion}

    In this experiment, characteristics of n-channel MOSFET are investigated. 
    
    In first stage, the the relation between gate voltage and the drain current is observed. It is 
    seen that so long as the transistor remains in the saturation region, the drain current is a 
    quadratic function of the gate voltage, where it remains at 0 until the gate voltage exceeds a 
    certain level. That is, the zero bias threshold voltage, below which the electric field created 
    by the gate voltage is not enough to form the inversion layer. As $V_{GS}$ increases beyond this
    threshold, drain current starts to rise, following a parabola. Drain voltage has no effect in 
    this operation, so long as it is kept higher that the $V_{GS}+V_{th}$, otherwise transistor falls 
    out of the saturation and enters the triode region, where the drain current now is a linear function
    of the $V_{GS}$.

    In the second stage, relation between drain voltage and the drain current is observed under 
    various levels of gate voltage. It is seen that, for any gate voltage that doesn't put the 
    MOSFET into cut-off, $I_D$ rises almost linearly for relatively small $V_{DS}$. At this region,
    MOSFET behaves like a voltage controlled resistor whose value is inversly proportional to the 
    gate voltage. This relation however, for increasing $V_{DS}$, starts to lose its linearity.
    As the $V_{DS}$ exceeds $V_{GS}-V_{th}$, transistor enters the saturation region, where the 
    drain current remains constant, independent of the $V_{DS}$. Clearly, both the drain voltage 
    that puts the transistor into saturation $V_{DS_{(sat)}}$ and the drain current at the 
    saturation $I_{D_{(sat)}}$ increases with increasing $V_{GS}$.

    Lastly, a resistor in series with the gate is added into the circuit. It is seen that this resistor 
    had no effect whatsoever on the results of the simulation. However, since the gate behaves like a 
    small capacitor, when underwent transient simulation, it is seen that this resistor increases the 
    the time it takes gate to fully reach its maximum, hence delaying the peak of the drain current. 
    Approximate calculations estimated the value of the gate capacitance to be at the order of picofarads,
    hence quite small. However, at high frequencies, this capacitive behaviour can be very significant. 
    Of course, the MOSFET used for this part was a discrete MOSFET, and the MOSFETs in integrated 
    circuits may have drastically different values. 

    Effect of applying negative $V_{GS}$ is also studied at this part, while it appears to be no different 
    than applying any below-threshold voltage in DC analysis, it is seen that it can have a significant 
    effect in trainsient analysis. Capacitive nature of the gate results in diminishing of fast edges, 
    which consequently would deform any signal with high frequency components.

\end{document}